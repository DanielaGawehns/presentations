\documentclass [t,11pt] {beamer}
%\def\Put(#1,#2)#3{\leavevmode\makebox(0,0){\put(#1,#2){#3}}}

% --------------------------------------------------------------------
% Load packages
\usepackage{ulem}
\usepackage{graphicx}
\usepackage{tikz}
\usetikzlibrary{calc,trees,positioning,arrows,chains,shapes.geometric,%
    decorations.pathreplacing,decorations.pathmorphing,shapes,%
    matrix,shapes.symbols}
\usepackage{listings}
%\lstset{tabsize=2,showspaces=false,showtabs=false,basicstyle=\ttfamily\mdseries\itshape\normalsize}
%for listings:
\usepackage{color}

\usepackage{hyperref} %for links

\definecolor{dkgreen}{rgb}{0,0.6,0}
\definecolor{gray}{rgb}{0.5,0.5,0.5}
\definecolor{mauve}{rgb}{0.58,0,0.82}

\lstset{frame=tb,
  language=R,
  aboveskip=3mm,
  belowskip=3mm,
  showstringspaces=false,
  columns=flexible,
  basicstyle={\small\ttfamily},
  numbers=none,
  numberstyle=\tiny\color{gray},
  keywordstyle=\color{blue},
  commentstyle=\color{dkgreen},
  stringstyle=\color{mauve},
  breaklines=true,
  breakatwhitespace=true,
  tabsize=3
}





% --------------------------------------------------------------------
% Beamer version theme settings
%\usetheme[faculty=sciences,lang=en,rmfont=pmn,logofont=fpi]{leiden}
%\usetheme[faculty=sciences,lang=en]{leiden}
\usetheme[faculty=sciences,lang=en,logofont=fpi]{leiden}

% --------------------------------------------------------------------
\def\liketitle#1{%
{\usebeamerfont{frametitle}\usebeamercolor[fg]{frametitle}%
\begin{flushleft}%
\vspace{-\baselineskip}% Cosmetic correction for space introduced by flushleft
#1\par
\end{flushleft}%
\vspace{-\baselineskip}% Cosmetic correction for space introduced by flushleft
}%
\vspace{0.75\baselineskip}%
}

\setbeamertemplate{navigation symbols} {}

%\usetheme{Singapore}

% Header settings
\def\lecturename{Digital OS} 
\lecture[Leiden Template]{}{ldn-bmr}
%\lecture[Leiden Template]{}{}
\title{Journal Club: Tools for Digital Open Science} 
%\subtitle{Template to generate Leiden-style slides with LaTeX}
\author {Daniela Gawehns, ReproducibiliTea Leiden 1-10-2020}
\date{}
%\institute{Universiteit Leiden}
\subject{Lecture: \lecturename}


\begin{document}


\maketitle


%%%%%%%%%%%%%%%%%

\begin {frame} {Paper}


\begin {center }

\href{https://journals.plos.org/plosbiology/article?id=10.1371/journal.pbio.2006022}{Digital open science - \\Teaching digital tools for reproducible and transparent research}\\
 \end {center}
Toelch U, Ostwald D (2018)

\end {frame}

%%%%%%%%%%%%%%%%%

\begin {frame} { Course Content }

The Course: 
\begin {itemize}
\item Session I  Introduction to open science
\item Session II Preregistration
\item Session III Data Repositories (OSF, Dataverse, Zenodo, FAIR data)
\item Session IV Code Management (Git, Notebooks)
\item Session V Open Access (publication process, preprints, licencing)
\item Session VI Chances and Limitations of Open Science
\item Course Project: Implement what your learnt
\end {itemize}


\end {frame}

%%%%%%%%%%%%%%%%%


\begin {frame} {Course Context}

MSc of PhD level course \\
Life Science as main target audience \\
60 hrs total with 15-20 hrs lectures/tutorials \\
\\
Narrative: Replicate a study from your field of interest

\end {frame}

%%%%%%%%%%%%%%%%%

\begin {frame} { Final Course Project}

Aim: Integrate what you learnt into your workflow

\end {frame}

%%%%%%%%%%%%%%%%%

\begin {frame} {Course Evaluation and Tips}
\begin {itemize}
\item tailor course to scientific fields/ institutes
\item 1/3 of respondents had NO plans to use version control in their projects
\item incorporate in existing coursework (either content or ethics or data management course)
\end {itemize}

\end {frame}
%%%%%%%%%%%%%%%%%

\begin {frame} {What's next?}

Discussion Questions:
\begin {itemize}
\item Do you know a (mandatory?) course where open science content could be added?
\item What about the Bachelor Students?
\item What do you think about field-specific courses?
\item What about the argument that a reproducible workflow makes your own life easier?
\item When is the best time to learn about OS tools?
\end {itemize}

\end {frame}
%%%%%%%%%%%%%%%%%





\end {document}